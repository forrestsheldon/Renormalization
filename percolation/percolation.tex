\input{header.tex}
\title{Percolation in Random Resistor Networks}
\author{Forrest Sheldon}
\date{\today}
\begin{document}
\maketitle

\section{Introduction}

Percolation deals with the properties of clusters created by some
probabilistic means. Generally, past a threshold an infinite cluster
will form that spans the lattice and we consider this a phase transition.
While the process that generates this cluster is simply stated, its properties
are quite rich, resisting analytic treatment and exhbiting complex phenomena
such as multifractal scaling. The marriage of conceptual simplicity and
complex behavior makes percolation a popular introduction to phase transitions,
and the generality of the model makes it applicable to a wide variety of physical
problems from superconductivity to forest fires.


This coverage aims to give an introduction to percolation theory with particular
attention to the problem as a conduction transition in a random resistor network.
We will follow our coverage of the Ising model, first introducing percolation and
giving brief attention to it's variants and correspondence to physical transitions.
Then we will
review exact solutions on simple lattices, namely in one dimension and the Bethe
lattice.  Next we will review mean field theory techniques away from the transition
and then scaling relations derivable from assumptions about the form of the
infinite cluster.  Finally, we will review applications of renormalization, and give
a summary of extensions to more complex examples.

\subsection{The Basic Percolation Problem}

Consider a finite cubic lattice where neighboring sites are not yet connected.
At every potential edge, place a bond with a probability, or \textbf{concentration}
$p$. As we increase this concentration from 0, at some point instantiations of the
lattice will contain clusters that connect from the top to the bottom and we say
the lattice \textbf{percolates}.  As the size of the lattice
increases, the point at which a percolating cluster forms becomes more sharply
defined and in the infinite limit we can define a critical concentration
$p_c$ beyond which an infinite cluster exists.

We may consider a similar 

\subsubsection{Bond vs. Site Percolation}
\subsection{Critical Exponents}
\subsection{Physical Correspondences}
\section{Exact Solutions}
\subsection{One-Dimension}
\subsection{Bethe Lattice}
\section{Mean-Field Techniques}
\subsection{Mean Field Theory}
\subsection{Effective Medium Theory}
\section{Scaling Relations}
\section{Renormalization}
\section{Extensions}
\end{document}
