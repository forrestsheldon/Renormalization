% ***********************************************************
% ******************* PHYSICS HEADER ************************
% ***********************************************************
% Version 2
\documentclass[11pt]{article} 
\usepackage{amsmath} % AMS Math Package
\usepackage{amsthm} % Theorem Formatting
\usepackage{amssymb}	% Math symbols such as \mathbb
\usepackage{graphicx} % Allows for eps images
\usepackage{multicol} % Allows for multiple columns
\usepackage[dvips,letterpaper,margin=1in,top=1in,bottom=1in]{geometry}
\renewcommand{\labelenumi}{(\alph{enumi})} % Use letters for enumerate
% \DeclareMathOperator{\Sample}{Sample}
\let\vaccent=\v % rename builtin command \v{} to \vaccent{}
\renewcommand{\v}[1]{\ensuremath{\mathbf{#1}}} % for vectors
\newcommand{\gv}[1]{\ensuremath{\mbox{\boldmath$ #1 $}}} 
% for vectors of Greek letters
\newcommand{\uv}[1]{\ensuremath{\mathbf{\hat{#1}}}} % for unit vector
\newcommand{\abs}[1]{\left| #1 \right|} % for absolute value
\newcommand{\avg}[1]{\left< #1 \right>} % for average
\let\underdot=\d % rename builtin command \d{} to \underdot{}
\renewcommand{\d}[2]{\frac{d #1}{d #2}} % for derivatives
\newcommand{\dd}[2]{\frac{d^2 #1}{d #2^2}} % for double derivatives
\newcommand{\pd}[2]{\frac{\partial #1}{\partial #2}} 
% for partial derivatives
\newcommand{\pdd}[2]{\frac{\partial^2 #1}{\partial #2^2}} 
% for double partial derivatives
\newcommand{\pdc}[3]{\left( \frac{\partial #1}{\partial #2}
 \right)_{#3}} % for thermodynamic partial derivatives
\newcommand{\ket}[1]{\left| #1 \right>} % for Dirac bras
\newcommand{\bra}[1]{\left< #1 \right|} % for Dirac kets
\newcommand{\braket}[2]{\left< #1 \vphantom{#2} \right|
 \left. #2 \vphantom{#1} \right>} % for Dirac brackets
\newcommand{\matrixel}[3]{\left< #1 \vphantom{#2#3} \right|
 #2 \left| #3 \vphantom{#1#2} \right>} % for Dirac matrix elements
\newcommand{\grad}[1]{\gv{\nabla} #1} % for gradient
\let\divsymb=\div % rename builtin command \div to \divsymb
\renewcommand{\div}[1]{\gv{\nabla} \cdot #1} % for divergence
\newcommand{\curl}[1]{\gv{\nabla} \times #1} % for curl
\let\baraccent=\= % rename builtin command \= to \baraccent
\renewcommand{\=}[1]{\stackrel{#1}{=}} % for putting numbers above =
\newtheorem{prop}{Proposition}
\newtheorem{thm}{Theorem}[section]
\newtheorem{lem}[thm]{Lemma}
\theoremstyle{definition}
\newtheorem{dfn}{Definition}
\theoremstyle{remark}
\newtheorem*{rmk}{Remark}

% ***********************************************************
% ********************** END HEADER *************************
% ***********************************************************

\title{Percolation in Random Resistor Networks}
\author{Forrest Sheldon}
\date{\today}
\begin{document}
\maketitle

\section{Introduction}

Percolation deals with the properties of clusters created by some
probabilistic means. Generally, past a threshold an infinite cluster
will form that spans the lattice and we consider this a phase transition.
While the process that generates this cluster is simply stated, its properties
are quite rich, resisting analytic treatment and exhbiting complex phenomena
such as multifractal scaling. The marriage of conceptual simplicity and
complex behavior makes percolation a popular introduction to phase transitions,
and the generality of the model makes it applicable to a wide variety of physical
problems from superconductivity to forest fires.


This coverage aims to give an introduction to percolation theory with particular
attention to the problem as a conduction transition in a random resistor network.
We will follow our coverage of the Ising model, first introducing percolation and
giving brief attention to it's variants and correspondence to physical transitions.
Then we will
review exact solutions on simple lattices, namely in one dimension and the Bethe
lattice.  Next we will review mean field theory techniques away from the transition
and then scaling relations derivable from assumptions about the form of the
infinite cluster.  Finally, we will review applications of renormalization, and give
a summary of extensions to more complex examples.

\subsection{The Basic Percolation Problem}

Consider a finite cubic lattice where neighboring sites are not yet connected.
At every potential edge, place a bond with a probability, or \textbf{concentration}
$p$. As we increase this concentration from 0, at some point instantiations of the
lattice will contain clusters that connect from the top to the bottom and we say
the lattice \textbf{percolates}.  As the size of the lattice
increases, the point at which a percolating cluster forms becomes more sharply
defined and in the infinite limit we can define a critical concentration
$p_c$ beyond which an infinite cluster exists.

We may consider a similar problem where bonds are all occupied but \emph{sites}
are initially vacant.  Each site is occupied with a probability $p$ and clusters
are collections of occupied sites connected by bonds.  This defines the
\textbf{site percolation} as opposed to \textbf{bond percolation} mentioned previously.
Depending on the physical problem we are interested in, we may choose to frame
it in terms of sites or bonds but the results given by either will be closely
related.  Here we will focus on bond percolation with the intention of relating our
results to the random resistor network.  This coverage is adapted from \cite{stauffer94}
\cite{kristenson} which both cover site percolation.  Results here have been
rederived for bond percolation following their arguments.

\subsection{Important Quantities and Exponents}

We will make use of several quantities to characterize the lattice and it's transition.
Here we give a brief summary of each, important relations
involving them and their critical exponents:

\begin{enumerate}

\item[$p_c$] The \textbf{percolation threshold} is the concentration at which an infinite cluster
will form on an infinite lattice.  While easily evaluated on the Bethe lattice, analytical
results resisted attempts for 20 years after the problem was initally posed.  As with
the critical temperature, it is nonuniversal and depends on the details of the lattice
structure.  Currently accepted values may be viewed in Table \ref{tab:perc_thresh}.

\begin{table}
  \centering
  \begin{tabular}{ | l | r | l | l | }
    \hline
    Lattice & Coord. \# & Site & Bond \\
    \hline \hline
    1d & 2 & 1 & 1 \\ \hline
    2d Honeycomb & 3 & 0.6962 & $1-2\sin{\pi/8}\approx 0.65271$ \\ \hline
    2d Square & 4 & 0.592746 & 1/2 \\ \hline
    2d Triangular & 6 & 1/2 & $2\sin{\pi/18}\approx 0.34729$ \\ \hline
    3d Diamond & 4 & 0.43 & 0.388 \\ \hline
    3d Simple cubic & 6 & 0.3116 & 0.2488 \\ \hline
    4d Hypercubic & 8 & 0.197 & 0.1601 \\ \hline
    5d Hypercubic & 10 & 0.141 & 0.1182 \\ \hline
    Bethe Lattice & z & 1 / (z-1) & 1 / (z-1) \\ \hline
  \end{tabular}
  \caption{Percolation thresholds for various lattices.  Note that within a dimension,
           the threshold decreases with increasing coordination number.  This is an
           abridged version of a table found in KRISTENSON.}
  \label{tab:perc_thresh}
\end{table}

\item[$n_s(p)$] The \textbf{cluster number distribution} gives the average number of
clusters of size $s$ per lattice site (such that in a lattice of size $N$ there will be on
average $Nn_s(p)$ clusters of size $s$ at concentration $p$).  As all bonds must belong
to a cluster of some size (at least below the percolation threshold) we have
\(\sum_s sn_s(p) = p\). This quantity $sn_s(p)$ is somewhat
analogous to a Boltzman weight in that it gives us information on the probabilities of
certain configurations of the system.

\item[$s_\xi$] We will find that the cluster number
distribution away from the percolation threshold, $n_s \propto \exp{(s/s_\xi)}$.
The scaling factor in the denominator is the \textbf{cutoff cluster size}.  This will
generally diverge at the percolation threshold allowing us to define the critical exponent
$\sigma$
\[ s_\xi \propto |p_c - p|^{-\frac{1}{\sigma}} \ \text{as}\ p\to p_c \]

\item[$S$] The cluster number distribution allows us to calculate a \text{mean cluster
size} which we define as the average cluster size to which a given occupied bond belongs.
As the probability of occupied sites belonging to s clusters is
\(\frac{sn_s(p)N}{pN} = \frac{sn_s(p)}{\sum sn_s(p)}\) the average cluster size is
\[S(p) = \frac{\sum s^2 n_s(p)}{\sum s n_s(p)}\]
(Note that in this definition clusters are sampled by \emph{site} biasing our selection
towards larger clusters.  We could similarly have defined an average cluster size in which
all clusters are sampled with equal probablity
\(S'(p) = \frac{\sum s n_s(p)}{\sum  n_s(p)}\) but we will find that our current definition
is more appropriate.)

The mean cluster size will also diverge at the percolation threshold, allowing us to
define the critical exponent $\gamma$
\[ S \propto |p_c - p|^{-\gamma} \ \text{as}\ p\to p_c \]


\end{enumerate}

\subsection{Critical Exponents}
\subsection{Physical Correspondences}
\section{Exact Solutions}
\subsection{One-Dimension}
\subsection{Bethe Lattice}
\section{Mean-Field Techniques}
\subsection{Mean Field Theory}
\subsection{Effective Medium Theory}
\section{Scaling Relations}
\section{Renormalization}
\section{Extensions}
\end{document}
