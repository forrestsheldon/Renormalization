% ***********************************************************
% ******************* PHYSICS HEADER ************************
% ***********************************************************
% Version 2
\documentclass[11pt]{article} 
\usepackage{amsmath} % AMS Math Package
\usepackage{amsthm} % Theorem Formatting
\usepackage{amssymb}	% Math symbols such as \mathbb
\usepackage{graphicx} % Allows for eps images
\usepackage{multicol} % Allows for multiple columns
\usepackage[dvips,letterpaper,margin=1in,top=1in,bottom=1in]{geometry}
\renewcommand{\labelenumi}{(\alph{enumi})} % Use letters for enumerate
% \DeclareMathOperator{\Sample}{Sample}
\let\vaccent=\v % rename builtin command \v{} to \vaccent{}
\renewcommand{\v}[1]{\ensuremath{\mathbf{#1}}} % for vectors
\newcommand{\gv}[1]{\ensuremath{\mbox{\boldmath$ #1 $}}} 
% for vectors of Greek letters
\newcommand{\uv}[1]{\ensuremath{\mathbf{\hat{#1}}}} % for unit vector
\newcommand{\abs}[1]{\left| #1 \right|} % for absolute value
\newcommand{\avg}[1]{\left< #1 \right>} % for average
\let\underdot=\d % rename builtin command \d{} to \underdot{}
\renewcommand{\d}[2]{\frac{d #1}{d #2}} % for derivatives
\newcommand{\dd}[2]{\frac{d^2 #1}{d #2^2}} % for double derivatives
\newcommand{\pd}[2]{\frac{\partial #1}{\partial #2}} 
% for partial derivatives
\newcommand{\pdd}[2]{\frac{\partial^2 #1}{\partial #2^2}} 
% for double partial derivatives
\newcommand{\pdc}[3]{\left( \frac{\partial #1}{\partial #2}
 \right)_{#3}} % for thermodynamic partial derivatives
\newcommand{\ket}[1]{\left| #1 \right>} % for Dirac bras
\newcommand{\bra}[1]{\left< #1 \right|} % for Dirac kets
\newcommand{\braket}[2]{\left< #1 \vphantom{#2} \right|
 \left. #2 \vphantom{#1} \right>} % for Dirac brackets
\newcommand{\matrixel}[3]{\left< #1 \vphantom{#2#3} \right|
 #2 \left| #3 \vphantom{#1#2} \right>} % for Dirac matrix elements
\newcommand{\grad}[1]{\gv{\nabla} #1} % for gradient
\let\divsymb=\div % rename builtin command \div to \divsymb
\renewcommand{\div}[1]{\gv{\nabla} \cdot #1} % for divergence
\newcommand{\curl}[1]{\gv{\nabla} \times #1} % for curl
\let\baraccent=\= % rename builtin command \= to \baraccent
\renewcommand{\=}[1]{\stackrel{#1}{=}} % for putting numbers above =
\newtheorem{prop}{Proposition}
\newtheorem{thm}{Theorem}[section]
\newtheorem{lem}[thm]{Lemma}
\theoremstyle{definition}
\newtheorem{dfn}{Definition}
\theoremstyle{remark}
\newtheorem*{rmk}{Remark}

% ***********************************************************
% ********************** END HEADER *************************
% ***********************************************************

\title{Percolation in Random Resistor Networks}
\author{Forrest Sheldon}
\date{\today}
\begin{document}
\maketitle

\section{Introduction}

Percolation deals with the properties of clusters created by some
probabilistic means. Generally, past a threshold an infinite cluster
will form that spans the lattice and we consider this a phase transition.
While the process that generates this cluster is simply stated, its properties
are quite rich, resisting analytic treatment and exhbiting complex phenomena
such as multifractal scaling. The marriage of conceptual simplicity and
complex behavior makes percolation a popular introduction to phase transitions,
and the generality of the model makes it applicable to a wide variety of physical
problems from superconductivity to forest fires.


This coverage aims to give an introduction to percolation theory with particular
attention to the problem as a conduction transition in a random resistor network.
We will follow our coverage of the Ising model, first introducing percolation and
giving brief attention to it's variants and correspondence to physical transitions.
Then we will
review exact solutions on simple lattices, namely in one dimension and the Bethe
lattice.  Next we will review mean field theory techniques away from the transition
and then scaling relations derivable from assumptions about the form of the
infinite cluster.  Finally, we will review applications of renormalization, and give
a summary of extensions to more complex examples.

\subsection{The Basic Percolation Problem}

Consider a finite cubic lattice where neighboring sites are not yet connected.
At every potential edge, place a bond with a probability, or \textbf{concentration}
$p$. As we increase this concentration from 0, at some point instantiations of the
lattice will contain clusters that connect from the top to the bottom and we say
the lattice \textbf{percolates}.  As the size of the lattice
increases, the point at which a percolating cluster forms becomes more sharply
defined and in the infinite limit we can define a critical concentration
$p_c$ beyond which an infinite cluster exists.

We may consider a similar problem where bonds are all occupied but \emph{sites}
are initially vacant.  Each site is occupied with a probability $p$ and clusters
are collections of occupied sites connected by bonds.  This defines the
\textbf{site percolation} as opposed to \textbf{bond percolation} mentioned previously.
Depending on the physical problem we are interested in, we may choose to frame
it in terms of sites or bonds but the results given by either will be closely
related.  Here we will focus on bond percolation with the intention of relating our
results to the random resistor network.  This coverage is adapted from \cite{stauffer94}
\cite{christensen02} which both cover site percolation.  Results here have been
rederived for bond percolation following their arguments.

%==================================================================================
% IMPORTANT QUANTITIES AND EXPONENTS
%==================================================================================

\subsection{Important Quantities and Exponents}

We will make use of several quantities to characterize the lattice and it's transition.
Here we give a brief summary of each, important relations
involving them and their critical exponents:

\begin{enumerate}

\item[$p_c\quad$] The \textbf{percolation threshold} is the concentration at which an infinite cluster
will form on an infinite lattice.  While easily evaluated on the Bethe lattice, analytical
results resisted attempts for 20 years after the problem was initally posed.  As with
the critical temperature, it is nonuniversal and depends on the details of the lattice
structure.  Currently accepted values may be viewed in Table \ref{tab:perc_thresh}.

\begin{table}
  \centering
  \begin{tabular}{ | l | r | l | l | }
    \hline
    Lattice & Coord. \# & Site & Bond \\
    \hline \hline
    1d & 2 & 1 & 1 \\ \hline
    2d Honeycomb & 3 & 0.6962 & $1-2\sin{\pi/8}\approx 0.65271$ \\ \hline
    2d Square & 4 & 0.592746 & 1/2 \\ \hline
    2d Triangular & 6 & 1/2 & $2\sin{\pi/18}\approx 0.34729$ \\ \hline
    3d Diamond & 4 & 0.43 & 0.388 \\ \hline
    3d Simple cubic & 6 & 0.3116 & 0.2488 \\ \hline
    4d Hypercubic & 8 & 0.197 & 0.1601 \\ \hline
    5d Hypercubic & 10 & 0.141 & 0.1182 \\ \hline
    Bethe Lattice & z & $1 / (z-1)$ & $1 / (z-1)$ \\ \hline
  \end{tabular}
  \caption{Percolation thresholds for various lattices.  Note that within a dimension,
           the threshold decreases with increasing coordination number.  This is an
           abridged version of a table found in \cite{christensen02}.}
  \label{tab:perc_thresh}
\end{table}

\item[P(p)] The \textbf{strength} of the infinite cluster gives the probability that
an arbirary bond is connected to the infinite cluster.  As such, $P$ is 0 below
the percolation threshold and goes to 1 as $p\to 1$.  In most lattices, at $p_c$
$P$ increases continuously from 0 and so is analogous to our order parameter for a second
order phase transition.  This allows us to define the critical exponent $\beta$,
\[ P(p) \propto (p - p_c)^{\beta} \ \text{as}\ p\to p_c^+ \]

\item[$n_s(p)$] The \textbf{cluster number distribution} gives the average number of
clusters of size $s$ per lattice site (such that in a lattice of size $N$ there will be on
average $Nn_s(p)$ clusters of size $s$ at concentration $p$).  As all bonds must belong
to a cluster of some size, we have,
\begin{align*}
\sum_s sn_s(p) = p,&\quad p<p_c \\
\sum_s sn_s(p)+ P(p) = p,&\quad p>p_c \\
\end{align*}
 where the sum excludes the infinite cluster. This
quantity $sn_s(p)$ is somewhat
analogous to a Boltzman weight in that it gives us information on the probabilities of
certain configurations of the system.

\item[$s_\xi\quad$] We will find that the cluster number
distribution away from the percolation threshold, $n_s \propto \exp{(s/s_\xi)}$.
The scaling factor in the denominator is the \textbf{cutoff cluster size}.  This will
generally diverge at the percolation threshold allowing us to define the critical exponent
$\sigma$
\[ s_\xi \propto |p_c - p|^{-\frac{1}{\sigma}} \ \text{as}\ p\to p_c \]

\item[$S$\quad] The cluster number distribution allows us to calculate a \text{mean cluster
size} which we define as the average cluster size to which a given occupied bond belongs.
As the fraction of occupied bonds belonging to s clusters is
\(\frac{sn_s(p)N}{pN} = \frac{sn_s(p)}{\sum sn_s(p)}\) the average cluster size is
\[S(p) = \frac{\sum s^2 n_s(p)}{\sum s n_s(p)}\]
(Note that in this definition clusters are sampled by \emph{bond} biasing our selection
towards larger clusters.  We could similarly have defined an average cluster size in which
all clusters are sampled with equal probablity
\(S'(p) = \frac{\sum s n_s(p)}{\sum  n_s(p)}\) but we will find that our current definition
is more appropriate.) This is analogous to our susceptibility in thermal problems.

The mean cluster size will also diverge at the percolation threshold, allowing us to
define the critical exponent $\gamma$
\[ S \propto |p_c - p|^{-\gamma} \ \text{as}\ p\to p_c \]

\item[$g(r)$] The \textbf{correlation function} or \textbf{pair connectivity} is the
probability that a bond a distance $r$ away from an occupied site belongs to the same
cluster.  In analogy to the susceptibilty sum rule, we have a relation between the
correlation function and the mean cluster size,
\[\sum_{\vec{r}} g(\vec{r}) = S\]
which is our motivation for defining the average cluster size as we have.

\item[$\xi\quad$] We will find that we can typically write the correlation function as
\(g(r) \propto \exp{(r/\xi)}\) where the scaling factor $\xi$ is the
\textbf{correlation length}. It will typically diverge at the percolation threshold,
allowing us to define the critical exponent $\nu$,
\[ \xi \propto |p_c - p|^{-\nu} \ \text{as}\ p\to p_c \]
\end{enumerate}

%==================================================================================
% EXACT SOLUTIONS
%==================================================================================

\section{Exact Solutions}

In this section we will calculate the preceding quantities in two scenarios in which they
may be evaluated exactly: the one dimensional chain and the Bethe Lattice.  Again
similar coverages for site percolation (which is identical to bond percolation in 1d)
may be found in \cite{stauffer94}\cite{christensen02}

\subsection{One-Dimension}

On an infinite lattice, any unoccupied bond will break the infinite cluster and thus
we can only have percolation at $p_c = 1$ (in analogy with the 1d Ising model).  The
strength $P(p)$ is thus zero for $p<1$ and jumps discontinuously to $1$ at percolation.

To find the cluster number distribution, we note that the probability for a cluster
of size $s$ to lie at any given position is $p^s(1-p)^2$.  As the number of possible
locations for such a cluster is equal to the number of bonds we have (from the linearity
of the expectation) that the expected number of such clusters \emph{per bond} is
\[n_s(p) = p^s(1-p)^2\]
As a check, we can evaluate,
\begin{align*}
\sum_s sn_s(p) &= \sum_s sp^s(1-p)^2 \\
 &= (1-p)^2 (p\d{}{p})\sum_s p^s \\
 &= p\frac{(1-p)^2}{(1-p)^2} = p \\
\end{align*}
We may write this
quantity as,
\[n_s(p) = (p_c - p)^2\exp{(-s/s_\xi)}, \quad s_\xi = -\frac{1}{\ln{p}}\]
and obtain the cutoff cluster size.  We thus have
\[s_\xi = -\frac{1}{\ln(1 - (p_c - p))}\propto |p_c - p|^{-1}\]
yielding the critical exponent $\sigma=1$.

To evaluate the average cluster size, we proceed as with our previous check on
the cluster number distribution to obtain,
\[S = \frac{\sum_s s^2n_s(p)}{\sum_s sn_s(p)} = \frac{1+p}{1-p}\]
and we can immediately identify $\gamma=1$.

Lastly, in order for two bonds separated by a distance $r$ to be connected, all
bonds between them must be connected and thus \(g(r) = p^r = \exp(-r/\xi)\) giving
\[\xi = -\frac{1}{\ln(1 - (p_c - p))}\propto |p_c - p|^{-1}\]
and $\nu = 1$.  We can confirm the susceptibility sum rule noting that the sum over
$\vec{r}$ breaks into a contribution from $r=0$ and two from bonds to the left and
right.
\[\sum_{\vec{r}}g(r) = 1 + 2\sum_{r=1,2..} p^r = 
1 + 2p\sum_{s=0,1,2..} p^s= \frac{1+p}{1-p} = S\]

\subsection{Bethe Lattice/Mean-Field Solution}

The Bethe lattice is an infinite tree (read no loops) in which each site has
$z$ nearest neighbors.  If $z=2$ we regain the 1d chain. The Bethe lattice is 
so named for the Bethe Peierls
approximation which is exact upon it and the ease of solution on the Bethe lattice stems
from the fact that every pair of sites possesses a single path joining them and its
translation symmetry may be used to obtain self consistency equations for many
quantities. Readers interested in its history and breadth of application in physics
can consult Thorpe's charming and seemingly obscure coverage in \cite{thorpe82}.

Considering entering a site through an occupied bond, we expect that the lattice will
percolate when the expected number of bonds continuing onward is 1.  This yields the
correct percolation threshold of $p_c = \frac{1}{z-1}$ which may be established
rigorously \cite{fisher61}.

Our argument for the cluster number distribution in 1d applies here as well,
but suffers from a complication.  In 1d, there is a single possible
configuration for a cluster of $s$ bonds: a chain of length $l$.  Such a chain
has a perimeter of 2 and so the contribution to the cluster number
distribution is of the form $g_{s,t}p^s(1-p)^t$ where $g_{s,t}$ is the
multiplicity of shapes of clusters of size $s$ and perimeter $t$.  We can
thus write the cluster number distribution as
\[n_s(p) = \sum_t g_{s,t}p^s(1-p)^t.\]


\section{Scaling Relations}
\section{Renormalization}
\section{Extensions}
\end{document}
