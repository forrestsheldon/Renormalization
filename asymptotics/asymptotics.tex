\input{header.tex}
\title{Renormalization and Singular Percolation Theory}
\author{Forrest Sheldon}
\date{\today}
\begin{document}
\maketitle

\section{Dimensional Analysis is Incredibly Useful}

The Buckingham Pi Theorem \cite{Buckingham1914} asserts that any equation
that completely describes
the relation between a collection of physical quantities takes the form,
\[\Pi = f( \Pi_0, \Pi_1, ...,\Pi_n) = 0\]
where the $\Pi_i$s are all the independent dimensionless products that may be
formed from the given quantities.  As an example of it's utility, consider the
diffusion equation in one dimension,
\[\partial_t u(x, t) = \frac{1}{2} \kappa \partial_{xx} u(x, t)\]
with an initial condition $u(x, 0) = \frac{A_0}{\sqrt{2\pi l^2}} e^{-x^2 / 2l^2}$.
By forming the dimensionless quantities,
\[\Pi = \frac{u \sqrt{\kappa t}}{A_0}\quad \Pi_1 = \frac{x}{\sqrt{\kappa t}} \quad
\Pi_2 = \frac{l}{\sqrt{\kappa t}} \]
we may immediately infer a solution of the form,
\[u  = \frac{A_0}{\sqrt{\kappa t}} f\left(\frac{x}{\sqrt{\kappa t}}, \frac{l}{\sqrt{\kappa t}}\right) \]
which forms a starting point for futher inquiry.

\subsection{Similarity Solutions}
Once the dimensional form of the equation is established, it is often productive to
examine similarity solutions.  In these functions, the arguments occur such that that length
and time scales are interdependent. (As an example $u(x, t) = t^\alpha f(x t^\beta).$)
By reducing the number of arguments of the scaling function $f$, similarity solutions
often allow us to convert PDEs to ODEs greatly facilitating their solution.  In addition,
the long term behavior of a system is often given by similarity solutions which may be
evidence of natural stabilization \cite{Barrenblatt1996}.  We call this regime, where
times/distances are large enough that boudary/initial values no longer influence the system
but the system is still far from equilibrium \textbf{intermediate asymptotics}.
\end{document}
