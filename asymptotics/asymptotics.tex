\input{header.tex}
\bibliographystyle{plain}
\title{Renormalization in Intermediate Asymptotics}
\author{Forrest Sheldon}
\date{\today}
\begin{document}
\maketitle

Following the discussion in Goldenfeld's text and a series of his papers, I have taken a
didactive approach to composing this paper, focusing on the logical progression of ideas 
allows renormalisation to be applied to dynamic phenomena.  Diffusion is used as the
'ising model of dynamic phenomena' to illustrate the emergence of anomalous dimensions and
the elucidation of their cause. Lastly renormalisation is used to tame the divergences
that arise in singular perturbation theory when a well behaved solution is known to exist.

\section{Dimensional Analysis is great!...when it works}

The Buckingham Pi Theorem \cite{Buckingham1914} asserts that any equation
that completely describes
the relation between a collection of physical quantities takes the form,
\[\Pi = f( \Pi_0, \Pi_1, ...,\Pi_n) = 0\]
where the $\Pi_i$s are all the independent dimensionless products that may be
formed from the given quantities.  As an example of it's utility, consider the
diffusion equation in one dimension,
\[\partial_t u(x, t) = \frac{1}{2} \kappa \partial_{xx} u(x, t)\]
with an initial condition $u(x, 0) = \frac{A_0}{\sqrt{2\pi l^2}} e^{-x^2 / 2l^2}$.
By forming the dimensionless quantities,
\[\Pi = \frac{u \sqrt{\kappa t}}{A_0}\quad \Pi_1 = \frac{x}{\sqrt{\kappa t}} \quad
\Pi_2 = \frac{l}{\sqrt{\kappa t}} \]
we may immediately infer a solution of the form,
\[u(x, t)  = \frac{A_0}{\sqrt{\kappa t}} f\left(\frac{x}{\sqrt{\kappa t}}, \frac{l}{\sqrt{\kappa t}}\right) \]
which forms a starting point for futher inquiry.

\subsection{Similarity Solutions}
Once the dimensional form of the equation is established, it is often productive to
examine similarity solutions.  In these functions, the arguments occur such that that length
and time scales are interdependent. (As an example $u(x, t) = t^\alpha f(x t^\beta).$)
By reducing the number of arguments of the scaling function $f$, similarity solutions
often allow us to convert PDEs to ODEs greatly facilitating their solution.  In addition,
the long term behavior of a system is often given by similarity solutions which may be
evidence of natural stabilization \cite{Barenblatt1996}.  We call this regime, where
times/distances are large enough that boudary/initial values no longer influence the system
but the system is still far from equilibrium \textbf{intermediate asymptotics}. (It is
worth noting that wave equations $f(x - vt)$ which also often describe long term behavior
may be written as a similarity solution.  Taking $x = \log X,\quad t = \log T$ the
solution becomes $f(\frac{X}{T^v})$.)

As an example of this procedure, we continue with the diffusion equation following
Goldenfeld \cite{Goldenfeld1992}.  In the long time limit, we expect the system
to be independent of $l$, the size of the initial state.  This suggests a similarity
solution,
\[u(x, t)  = \frac{A_0}{\sqrt{\kappa t}} f_s\left(\frac{x}{\sqrt{\kappa t}}\right) \propto t^{-1/2} f_s(\xi) \].
Substitution into the diffusion equation yields the ODE
\[f_s'' + \xi f' + f = 0\]
which is easily solved to give the solution in the intermediate regime,
$u(x, t) = \frac{A_0}{\sqrt{2\pi \kappa t}} e^{-x^2 / 2\kappa t}.$

We observe in this case that the system has no remaining dependence on the initial conditions
(other than it's parity and size $A_0$ which are conserved in its evolution).  By observing
this equation we cannot determine the initial size of the system, or the time at which the
system was begun and thus a locus of initial conditions all will progress to this form
in the intermediate limit.  This fact is the first hint that renormalization may have a
part to play in the analysis of systems of this type.  While unnecessary here, we will see
that for a variety of systems, renormalization techniques yield valuable insights and
numerics on dynamical phenomena evolving in time.  

\subsection{Intermediate Asymptotics}

In taking the limit of our diffusion equation solution  as $l \to 0$ we have made the
tacit assumption that the limit is regular.  In this case we were correct and the similarity
solution we obtained gave the correct asymptotic limit.  However, for a wide variety of
problems this will not be the case and taking the limit will give rise to anamolous dimensions.

Barenblatt \cite{Barenblatt1996} distinguishes three cases.  Given a relation between
dimensionless products $\Pi = f(\Pi_0, \Pi_1,...)$, as $\Pi_0 \to 0$,
\begin{enumerate}
\item $f$ approaches a finite, nonzero limit.  In this case, we may simply replace $\Pi_0$
with 0 and proceed with our analysis.  We call this \textbf{intermediate asymptotics of the
first kind}.
\item $f$ does not approach a finite limit but posseses power law asymptotics.
In this case, we may seek a solution of the form
\[\Pi = \Pi_0^\alpha g(\frac{\Pi_1}{\Pi_0^{\alpha_1}},\frac{\Pi_2}{\Pi_0^{\alpha_2}},...)\]
where the anomalous dimensions $\alpha, \alpha_1, ...$ must be determined.  Systems
displaying this behavior posses \textbf{intermediate asymptotics of the second kind}.
\item Neither of these two limits occur.
\end{enumerate}
As illustrated above, dimensional analysis may be capable of tackling problems of the first
kind.  For problems of the second type, we will need to invoke renormalisation to absorb
the singular limit.  To motivate this process, we follow another discussion from Goldenfeld
\cite{Goldenfeld1992}.

\section{Anomalous Dimensions in Problems of the Second Kind}

In this section we will examine a particular example of intermediate asymptotics of the 
second kind.  We will see that by allowing a normally conserved quantity to vary, we
introduce a singular dependence on the initial conditions in our scaling function.  Using
an insighful approximation, we will obtain the anomalous dimensions that arise as a
consequence without invoking renormalisation.

\subsection{Water Flow in Porous Media - Barenblatt's Equation}

Consider the half-space $z>0$ filled with a porous medium.  When water moves
into a region of this space, it occupies the maximum available space in each pore
$\sigma$ (a fraction of each pore is occupied by gas.).  When it exits an occupied
pore, adhesion causes it to leave a thin film in the pore such that $\sigma_0$
of the pore remains filled.  This suggests an asymmetry in our dynamics.  When
water flows into a space it requires a volume proportional to $\sigma$ to fill it,
but when exiting a space the amount of water that needs to flow out to consider that
space empty is proportional to $\sigma - \sigma_0$.  This suggests that when the
height of the water in a region is rising and thus the water is encountering new
pores, it diffuses with a slow diffusion constant $D$ but in regions where the height
is falling and less water must flow to empty the space, it diffuses at at some faster
rate with $(1 + \epsilon)D$.  These dynamics are contained in the Barenblatt equation which
we write in 1 dimension,
\[\partial_t \,u(x, t) = D\kappa\, \partial_{xx} u(x, t)\]
\[ D = \left\{
             \begin{array}{l l}
             \frac{1}{2} & \quad \partial_t \,u > 0 \\
             (1 + \epsilon)\frac{1}{2} &  \quad \partial_t \,u < 0
             \end{array} \right. \]

We denote the point between these two regimes as the positive $x$ value, $x_0(t, \epsilon)$
such that $\partial_t\, u(x_0, t) = 0$ It has been proved that solutions to this equation exist,
are unique, and possess smooth derivatives in $x$ and $t$.

\subsection{Attempting a Similarity Solution}

As with the diffusion equation, we form the dimensionless quantities
\[\Pi = \frac{u \sqrt{\kappa t}}{A_0}\quad \Pi_1 = \frac{x}{\sqrt{\kappa t}} \quad
\Pi_2 = \frac{l}{\sqrt{\kappa t}}\quad \Pi_3 = \epsilon\]
which suggest a similarity solution of the form,
\[u_s \propto t^{-1/2} f(\xi,\epsilon) \quad \xi = \frac{x}{\sqrt{\kappa t}}\]
with $\xi_0 = \frac{x_0}{\sqrt{\kappa t}}$.  Plugging this into the PDE gives two
ODEs,
\begin{align*}
(1+\epsilon)f_1'' + \xi f_1' + f_1 = 0& \quad 0 \le \xi < \xi_0 \\
f_3'' + \xi f_2' + f_2 = 0& \quad  \xi_0< \xi < \infty \\
\end{align*}
both of which may be integrated to give,
\begin{align*}
f_1 =& B_1 e^{-\xi^2 / 2(1+\epsilon)} \\
f_2 =& B_2 e^{-\xi^2 / 2}.
\end{align*}
This seems like a success until we attempt to enforce our matching conditions at $\xi_0$.  In
order to match the functions and their first derivatives we are forced to conclude
$B_1 = B_2 = 0$.  Taken with the existence theorem of the solution, we are forced to conclude
that no similarity solution of the form given above exists and this problem belongs to
intermediate asymptotics of the second kind.

\subsection{A Heuristic Derivation using the Similarity solution with Anomalous Dimensions}

We may define the 'mass' of our solution by $m(t) = \int dx u(x, t)$.  The diffusion equation
conserves mass and we may thus set it to a constant $m(t) = A_0$ however our discussion
of diffusion in porous media makes it clear that the mass should no longer be conserved.
To see this explicity we examine,
\begin{align*}
\partial_t m(t) =& \int dx D\kappa \partial_{xx} u \\
=& \int dx \kappa (\partial_x D) (\partial_x u)
\end{align*}
where the last step follows from integration by parts.  A constant $D$ will thus conserve mass.
For Barenblatt's equation, $\partial_x D = \frac{\epsilon}{2}(\delta(x + x_0) - \delta(x - x_0))$
giving $\partial_t m(t) = \epsilon\kappa\left.\partial_x u\right|_{x_0, t}$.


If the diffusion is very slow, we might expect that the form of the solution is similar
to what we would expect from regular diffusion but with a mass that varies with time,
\[u \propto \frac{m(t)}{\sqrt{2\pi (kt + l^2)}} \exp\left(\frac{-x^2}{2(kt + l^2)}\right). \]
This approximation gives $x_0(t) = \sqrt{kt + l^2}$. Using this to evaluate $\partial_t m(t)$,
we obtain
\[\partial_t m(t) = -\frac{\epsilon \kappa}{\sqrt{2\pi e}} \frac{m(t)}{kt+l^2}\]
which we integrate from $t=0$ to get
\[m(t) = \frac{m(0) l^{2\alpha}}{(kt + l^2)^\alpha} \quad \alpha = \frac{\epsilon}{\sqrt{2\pi e}}.\]
Inserting this into our supposed asymptotic form and setting all $l$s that do not give a singular
constribution to 0 yields the intermediate asymptotic form
\[u \propto \frac{m(0)l^{2\alpha}}{ (kt)^{1/2 + \alpha}} e^\frac{-x^2}{2kt}.\]

It is interesting to note that while $\alpha$ is a consequence of the presence of an
initial width, it's value is independent of that width confirming that all initial
conditions will evolve towards this form.  We also note that unlike diffusion, the Barenblatt
equation possesses no solution for $\delta$ function initial conditions for whom $l=0$.

\section{Renormalisation}

Taking the last section as a parable, we will attempt to recreate the result we found
previously without assuming the explicit form of a solution.  The renormalisation we apply
here will consist of two parts: first, by inserting a parameter in our scaling 
function we will move the singular behavior out of the function.  This will imply a relationship
between the scaling of our parameter, and the behavior of the solution.  By approximating the
solution, we will be able to extract the scaling behavior and calculate the anomalous dimensions.
This coverage roughly follows that found in Goldenfeld \cite{Goldenfeld1992} and 
\cite{Goldenfeld1989}, both of which discuss Barenblatt's equation.

\subsection{Mass Renormalization}

We begin with our scaling assumption from dimensional analysis,
\[u(x, t) = \frac{m_l}{\sqrt{\kappa t}} f(\xi, \eta, \epsilon)\]
where we have called $\xi = \frac{x}{\sqrt{\kappa t}}$ and  $\eta = \frac{l}{\sqrt{\kappa t}}$.
We are interested in the limit $\eta \to 0$ in which we know $f$ is singular.  In the last
section we touched on the scaling in the intermediate regime $m_l(0) \propto \eta^{-\alpha} m(t)$.
According to this relationship, there is no unique value $m_l(0)$ that leads to the currently
observed value $m(t)$.  Instead, for every $l$ there is a separate $m_l(0)$ that leads to the
the currently observed value.  It thus follows both that we may redefine the value of $m_l(0)$
however we wish so long as we retain $m(t)$, and that the intermediate asymptotic solution
corresponds to the $\eta\to 0$ $m_l(0) \to \infty$ limit in which the initial mass will absorb
the singular behavior in $f$.

To formalize this insight, we reason that as $m(t)$ and $m_l(0)$ have the same units, they must
be proportional to one another and introduce the constant of proportionality $m(t) = \frac{m_l}{Z}$
(where from now one we will write $m_l(0) = m_l$).  As $m(t)$ in the intermediate regime is
independent of $l$, $Z$ must cancel the dependence and thus $Z=Z(\frac{l}{\mu})$ where $\mu$ is
an arbitrary length scale chosen to make the ratio dimensionless.  Inserting this into our solution
gives 
\[u = \frac{m(t)Z}{\sqrt{\kappa t}} f\left(\xi, \eta, \epsilon \right).\]
where $Z$ is defined such that for small $l$,
\[Z\left(\frac{l}{\mu} \right) f\left(\frac{x}{\sqrt{\kappa t}}, \frac{l}{\sqrt{\kappa t}},\epsilon\right)
 = F\left(\frac{x}{\sqrt{\kappa t}}, \frac{\mu}{\sqrt{\kappa t}}, \epsilon\right) + O(l)\quad l\to 0.\]
If this tranformation is possible, we then reason that the solution cannot depend on the 
arbitrary length scale $mu$ that we introduced.  This gives the renormalization group equation,
\[\mu \d{u}{\mu} = 0\]
which, eliminating $m$ in favor of $Z$  is,
\begin{align*}
\mu \d{}{\mu} \frac{m_l}{Z\sqrt{\kappa t}}F(\xi, \sigma, \epsilon) &= 0, \quad \sigma = 
\frac{\mu}{\sqrt{\kappa t}}.\\
\d{\log Z}{\log \mu}F - \sigma \d{F}{\sigma} &= 0
\end{align*}
As $m_l$ diverges with $l^{-2\alpha}$ the assumption that our renormalization has absorbed
the divergence is
\[\d{\log Z}{\log \mu} = 2\alpha\]
using which we can solve the full equation to give,
\[u(x, t) = \frac{m_l}{\sqrt{\kappa t}}\left(\frac{l}{\mu}\right)^{2\alpha} 
\left(\frac{\mu}{\sqrt{\kappa t}}\right)^{2\alpha} G(\xi, \epsilon) \]

\subsection{Perturbative Renormalization}

We now undertake to perform actually use the RG we have defined to calculate the anomalous
dimension of Barenblatt's equation.  To do so we perform a perturbative expansion on Barenblatt's
equation and use our approximation in the scheme above.  In doing so we will find that
although the solution to the equation is guaranteed to be smooth, the terms in the perturbation
series will diverge.  By use of the arbitrary length scale $\mu$ we introduced above, we will be
able to continually push these divergences to higher order, signaling that our actual
solutions are well behaved.  This is an example of a singular perturbation, in which even small
additions to the PDE drastically change the behavior of the solution.  As this calculation involves
many steps, we will split it up into subsections.

\subsubsection{ Perturbative expansion of the solution}

We may write Barenblatt's equation as a diffusion equation with a small perturbing term (and 
rescaling $\kappa$ to 1 for convenience),
\[\left(\partial_t - \frac{1}{2} \partial_{xx}\right) u(x, t) = 
\frac{\epsilon}{2}\Theta(x_0(t) - |x|) \partial_{xx} u(x, t).\]
We attempt an asymptotic expansion of the solution,
\[u(x, t) = u_0 + u_1\epsilon + u_2 \epsilon^2 +...\]
Writing the equation abstractly as $L_D u = \epsilon L_P u$ where $L_D$ is the diffusion operator and
$L_P$ is the perturbing operator, we obtain equations for the first two terms,
\begin{align*}
L_D u_0 &= 0 \\
L_D u_1 &= L_P u_0.
\end{align*}
With a gaussian initial condition with standard deviation $l$, $u_0$ is just,
\[u_0(x, t) = \frac{m_0}{\sqrt{2\pi(t + l^2)}}\exp\left(\frac{-x^2}{2(t+l^2)} \right).\]
and the first order term follows from the green's function,
\begin{align*}
u_1(x, t) &= \frac{m_0}{2}\int_0^t ds \int_{-\infty}^\infty dy\, G(x-y, t-s)\Theta(y_0(t) - |y|) \partial_{yy} u(y, t) \\
u_1(x, t) &= \frac{m_0}{2}\int_0^t \frac{ds}{s+l^2}  \int_{-\sqrt{s+l^2}}^{\sqrt{s+l^2}} dy
\frac{\exp(-(x-y)^2/(2(t-s)))}{\sqrt{2\pi(t-s)}}\frac{\exp(-y^2/(2(s+l^2)))}{\sqrt{2\pi(s+l^2)}}
\left(\frac{y^2}{t+l^2} - 1\right)
\end{align*}
\subsubsection{Leading Divergence in $u_1$}

This expression can be greatly cleaned up by the subsitution, $w = \frac{y}{\sqrt{s+l^2}}$ giving
\[u_1(x, t) = \frac{m_0}{4\pi}\int_0^t \frac{ds}{s+l^2} \int_{-1}^{1} dw
\frac{e^{-\frac{(x-\sqrt{s+l^2}w)^2}{2(t-s)}}}{\sqrt{t-s}}e^{-w^2/2}
\left(w^2 - 1\right). \]
Examining this term for divergences, it appears that for $l\to 0$ the only divergence is a
logarithmic one at small s.  We may thus set $l$ to 0 in all terms past the $\frac{1}{s+l^2}$,
\[u_1(x, t) \approx \frac{m_0}{4\pi}\int_0^t \frac{ds}{s+l^2} \int_{-1}^{1} dw\,
\frac{e^{-\frac{(x-\sqrt{s}w)^2}{2(t-s)}}}{\sqrt{t-s}}e^{-w^2/2}
\left(w^2 - 1\right). \]
Similarly, as the divergence occurs for $s\to 0$, we may set all $s$ terms to the right of the
logarithmic divergence to 0, leaving the leading singular contribution,
\[u_1(x, t) \approx \frac{e^{-\frac{x^2}{2t}}}{\sqrt{t}}\frac{m_0}{4\pi}\int_0^t \frac{ds}{s+l^2} \int_{-1}^{1} dw\,
e^{-w^2/2}\left(w^2 - 1\right). \]
A substitution of $w^2 = t$ and integration by parts on the last integral yields
\[2\int_{0}^{1} dw\,e^{-w^2/2}\left(w^2 - 1\right) = -\frac{2}{\sqrt{e}} \]
and the integral over $s$ in the limit $l\to 0$ gives the logarithmic term,
\[\int_0^t \frac{ds}{s+l^2} = \log\left(\frac{t}{l^2}\right). \]
To first order in $\epsilon$ our solution is thus,
\[u(x, t)  = \frac{m_0}{\sqrt{2\pi t}} e^{\frac{-x^2}{2t}} \left[1 - 
\frac{\epsilon}{\sqrt{2\pi e}}\log\left(\frac{t}{l^2}\right) + O(\epsilon^2) \right] + O(l,\epsilon)\]

\subsubsection{Perturbative Renormalisation}
We expand $Z$ in $\epsilon$, $Z = 1 + \sum_n z_n(l/\mu) \epsilon^n$.  Recalling $m(0) = Zm(t)$,
we insert this into our solution to obtain, keeping terms to order $\epsilon$,
\begin{align*}
u(x, t) =& \frac{m(t)}{\sqrt{2\pi t}} e^{\frac{-x^2}{2t}} \left[1 + \epsilon z_1 +...\right]
\times \left[ 1  - \frac{\epsilon}{\sqrt{2\pi e}}\log\left(\frac{t}{l^2}\right) + O(\epsilon^2) \right] \\
=&\frac{m(t)}{\sqrt{2\pi t}} e^{\frac{-x^2}{2t}} \left[ 1 + \epsilon 
\left(z_1 - \frac{1}{\sqrt{2\pi e}} \log \left(\frac{t}{l^2}\right)\right) + ... \right]
\end{align*}
Evidently, by choosing $z_1 = \frac{1}{\sqrt{2\pi e}} \log \left(C_1 \frac{\mu^2}{l^2}\right)$ we
will cancel the singular dependence on $l$ in our solution.

\subsubsection{Calculating the Anomalous Dimension}

At this point we can use the previously defined relationship,
\(\d{\log Z}{\log \mu} = 2\alpha \) to calculate the anomalous dimension.
\[\log Z = \log\left(1 + \frac{\epsilon}{\sqrt{2\pi e}} \log \left(C_1\frac{\mu^2}{l^2}\right) + ... \right) \approx \frac{2\epsilon}{\sqrt{2\pi e}}\log \mu + ..\]
and we have confirmed our earlier heuristic derivation $\alpha = \frac{\epsilon}{\sqrt{2\pi e}} + O(\epsilon^2)$
without the use of an ad hoc solution form.

\section{Wrapping Up}

After seeing traditional dimensional analysis fail to give appropriate forms for solutions
of certain problems, we discovered that the problem was a singular limit in our scaling form.
By introducing a renormalization parameter and using it to absorb the dependence of our
troublesome parameter we are able to calculate the degree of the singularity and extract
the long term behavior of the system.  Here we see that the flow in time which takes many initial
conditions to single 'fixed point' asymptotic solutions is exactly analogous to the application
of renormalization to thermal problems.

\bibliography{asymptotics}
\end{document}
